\subsection{Part 1 -- Manipulating Strings with Basic Expressions}

\begin{eval}
For each of the following expressions, predict what you expect the value to be on the first line (discussing it with your partner), then evaluate the expression in the REPL and confirm or correct your prediction on the second line if necessary. 
\begin{sevalenum}
\item "1 2 3".length()
\evallinetwo
\item "cats".charAt(2)
\evallinetwo
\item "thought".indexOf('h')
\evallinetwo
\item "ad astra".substring(5,5)
\evallinetwo
\item "per" + "" + "" + "aspera"
\evallinetwo
\item "you know it's " + (1 + 1 == 2)
\evallinetwo
\item "hello   there".length()
\evallinetwo
\item ("number " + (7 + 5)).length()
\evallinetwo
\end{sevalenum}
\end{eval}



\initialbox

\subsection{Part 2 -- Manipulating Strings with More Complex Expressions}

\begin{eval}
For each of the following expressions, predict what you expect the value to be on the first line (discussing it with your partner), then evaluate the expression in the REPL and confirm or correct your prediction on the second line if necessary. Assume the following code has been executed:

\begin{code}
String message = "Hello there";
\end{code}

\begin{sevalenum}
\item "Hello".indexOf('h')
\evallinetwo
\item message.indexOf('h')
\evallinetwo
\item message.substring(0, message.length())
\evallinetwo
\item message.substring(message.indexOf(' '), message.indexOf(h))
\evallinetwo
\item message.substring(0, message.indexOf(message.charAt(7)))
\evallinetwo
\item "give me " + 5 + 5 + " bananas."
\evallinetwo
\item "I love the number " 3 + 9
\evallinetwo
\end{sevalenum}
\end{eval}

\begin{eval}
Each of the following expressions will result in an error of some sort. Briefly summarize the error output from the REPL, then  explain what the real problem is on the line below. Consider that the following code has been executed:

\begin{code}
String message1;
String message2 = "hello";
\end{code}

\begin{sevalenum}
\item message1.length()
\evallinetwo
\item message2.substring(0, 6)
\evallinetwo
\item message2.charAt(message2.indexOf('z'))
\evallinetwo
\item message2.substring(0, message.length())
\evallinetwo
\item message2.charAt(message2.length())
\evallinetwo
\end{sevalenum}
\end{eval}


\initialbox

\subsection{Part 3 -- Writing Basic Methods with Strings}

*** write a method called 'uMad' that returns true if there is an exclamation point in an input string (tell them to use indexOf)

http://codingbat.com/prob/p220491

*** write a method called firstNameOnly that returns a substring up to but not including the first space in a string (tell them to use one line with indexOf and substring)

http://codingbat.com/prob/p281900

*** update firstNameOnly to account for strings with no spaces

\initialbox


\subsection{Part 4 -- Processing Strings with Loops}

*** write uMad without using indexOf

*** write a method that counts the number of digits in a string

http://codingbat.com/prob/p217885

*** write a method that removes all the exclamation points from a string

http://codingbat.com/prob/p203598

\initialbox

