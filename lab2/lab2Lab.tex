\subsection{Part 1 -- Working with Casting}

We'll start with a mildly challenging and practical casting exercise.

\begin{exer}

Create a variable of type double named x. Initialize this variable to any value you want. If you were to cast 
this value to an integer by typing the following:

\begin{code}
(int)x
\end{code}

You would truncate the value. For instance, if you initialized x to 5.8, you would get 5. 

Find an expression that would round x to the nearest whole value, no matter what value x holds. Write your 
expression below:

\evalline

Test your expression out with multiple values of x to make sure it works!
\end{exer}


\initialbox


\subsection{Part 2 -- Calling Static Methods}

In lab 1 you developed a very simple algorithm (probably only a single expression) for cubing a number. The Java API provides methods which perform may of these basic algorithms, especially mathematical functions. 

\begin{exer}
Make a call to Math.pow to compute the cube of 43. Write your method call expression below. 

\evalline

Check it against your algorithm from lab 1 to make sure you called it correctly. 
\end{exer}

\begin{exer}
Assuming you still have the variable x, make another double variable called y. Assign the values 8 and 40 to them, respectively. Make a call to Math.abs to determine the difference between these two numbers. Write your method call expression below:

\evalline

Now swap the values in x and y and try again. You should get the same distance!

Try this again with at least 2 more sets of numbers. Use some negative numbers and some fractional numbers. Your expression should work in all of these cases. 
\end{exer}

In addition to the static methods in the Math class, there are also static fields you can use. Static fields are variables that you can access from the context of a class and are often used for important constant values. For example, Math.PI is the mathematical constant PI. Enter Math.PI in to the REPL to see what it evaluates to. 

\begin{exer}
Create a double variable named r and initialize it to the value 5. Develop a formula for the area of a circle using r as the radius and the constant Math.PI. Provide your formula below:

\evalline

Test it out with other values for the radius to confirm it works. 
\end{exer}

Now let's look at some method calls that may produce some unexpected results.

\begin{eval}

Enter each of the expressions / statements below in to the REPL. On the following line, record what happened in the REPL. Then, on the line following that, explain why you got that result. Many of these expressions / statements will result in 
an error. If you're unsure how to explain it, refer back to the prelab and/or ask the instructor for help.

\begin{sevalenum}

\item Math.squareRoot(4.0)

\evallinetwo

\item int a = Math.pow(5, 2);

\evallinetwo

\item double b = Math.sqrt(-2);

\evallinetwo

\item abs(4 - 2)

\evallinetwo

\end{sevalenum}
\end{eval}

Note that you can get the last example above to work with a special kind of import statement that allows you to drop the class name from the static methods you're calling. Try typing in the following statement:

\begin{code}
import static java.lang.Math.abs
\end{code}

Then try that last example again. This will work for any other static method you wish to use a shorthand for!

\initialbox


\subsection{Part 3 -- Calling Instance Methods}

Now we'll create some object values and call instance methods from them. We'll start with a new class you haven't seen in the prelab yet. 

Create a new Random object by defining a variable of type Random named gen. Initialize it by calling the constructor 
for the Random class with no arguments.  Try calling the instance method nextInt() with no parameters from the object instance you stored in gen. You should get a random integer value back. 

\begin{exer}
We want to use this call to generate a random dice value (that is a number between 1 and 6, inclusive). Devise an expression that will reliably give you a value in this range. Be sure to test it by calling it many times! You should never get a value less than 1 (including zero and including negative numbers) and you should never get a value greater than 6. Provide your expression below, once you're confident it's working (it will be moderately complex). 

\evalline

\end{exer}

There are more simple ways to get the same result with Random objects. Look up the documentation page for the Random class for more information on how to use it.

\initialbox


\subsection{Part 4 -- Working with Graphics}

Now we'll work with some more interesting objects. We'll create a JFrame and a Graphics object and do some drawing. First, you'll need to import these classes:

\begin{code}
import javax.swing.JFrame;
import java.awt.Graphics;
\end{code}

Next, Create a JFrame object in a variable called window. We'll also call its setSize method and show method which do pretty much what you would expect:

\begin{code}
JFrame window = new JFrame();
window.setSize(400, 300);
window.show();
\end{code}

Now we'll need to graph a graphics object from this window that we can use to do some drawing with:

\begin{code}
Graphics g = window.getGraphics();
\end{code}

There are many methods available in the Graphics class (check the documentation link on Moodle). Let's start with a simple method for drawing lines. We'll draw a rectangle using four instance method calls:

\begin{code}
g.drawLine(50,50,50,80);
g.drawLine(50,80,80,80);
g.drawLine(80,80,80,50);
g.drawLine(80,50,50,50);
\end{code}

Note that there are other methods that handle the tasks of drawing more complex shapes for us. For instance, we can draw an overlapping rectangle in one line:

\begin{code}
g.drawRect(70,70,20,20);
\end{code}

When you're done, you can blank out the window with the following call:

\begin{code}
window.repaint();
\end{code}

\begin{exer}
Refer back to either your or your partner's drawing algorithm from lab 1. Translate it in to instance method calls for your Graphics object and try to draw the image from lab 1 as best as you can (it doesn't need to be perfect). Write down your translated algorithm below:

\evallinesix

\end{exer}

Take a look through the documentation for the graphics class and play with some of the other methods available!

\initialbox

Once you've received all the required signatures for this lab, scan your 
lab sheet to a PDF file and turn it in for the lab \#2 assignment. 
Make sure both you and your partner both turn in your own lab sheets. 
